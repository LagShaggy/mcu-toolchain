\chapter{Summary, Conclusions \& Future Work}

In this thesis, different Linux distributions for IoT edge devices were compiled, with a proposed standardized architecture in the foreground. To achieve this a market analysis of IoT hardware was performed, a variety of tools and open source projects such as Buildroot and JuiceVM were explored and introduced, and Linux and U-Boot binaries were emulated on QEMU. Furthermore, it was attempted to flash Linux binaries onto an STM32L476G-Eval development board which was not successful due to lacking community support for the board, and running a RISC-V emulation on ESP-EYE, which was technically successful but unusable in practice.

The diverse IoT ecosystem and the lack of standardization paired with a manufacturer-driven industry, are the primary justifications for this thesis. The variety of IoT network protocols and OSs, in this these referred to as frameworks rather than OSs, only amplify this heterogeneity. We propose standardization of the OS layer for IoT edge devices, such that an abstraction of the diverse hardware is possible. This would have the effect that board-specific implementations, such as bare metal code, are not required, but that the user can choose from diverse solutions that all work upon the OS layer, therefore the application layer would gain portability.

The process of this thesis was to evaluate possible MCU candidates, that displayed promising qualities, such as sufficient RAM, a CPU with more than 40MHz clock speed while considering energy efficiency and cost. STM in particular showed desired qualities such as a healthy community and the relevant ARM Cortex-M architectures. At this point, contact with STM employees was established and they generously supplied us with two STM32L476G-Eval development boards. Only then did we explore the means to find the right implementation for the desired goal, Linux on MCUs. This thesis concludes that this bottom-up approach is fundamentally flawed, it is not necessarily the hardware that enables the use of Linux, but the community that adapts the kernel to the physical layer. By searching for compatible hardware first, work previously done by the open source community is omitted. The best possible approach to solve a given use case is to compile a list of MCUs that are supported by Linux and tools such as Buildroot, not to choose a board and realize that it is not supported. In such a case, the upfront investment, of porting Linux to a specific board is not worth it. Yet, the explorative nature of this thesis brought forward a different perspective, and different tools such as Buildroot were discovered and evaluated.

Buildroot proved to be an invaluable tool for embedded Linux, as it contains all necessary components such as libraries like the \code{uClibc} and U-Boot. Supporting multiple architectures and subsequently multiple boards, even more so than the mainline Linux kernel. It should further be evaluated with the right boards available. As we performed tests on the STM32L476G-Eval development board, which did not have a dedicated configuration for compilation, we had to adapt configurations of similar boards,  which ultimately did not work because the contrast was too large, as is common in IoT.

While the success of this thesis was evaluated on QEMU in section~\ref{success}, it does not support all the different boards that are available on the market, understandably so. As STM boards were not present but recommendations that similar boards could be emulated instead and that the architecture is similar, yet it did not work as intended. Another aspect that QEMU can not achieve is to verify performance on the target architecture since the underlying hardware is much more capable.

JuiceVM was successfully run, but in practice, it is not a viable or usable solution due to its long boot times. Yet it is a very interesting implementation, and potentially opens up the door to further studies of virtualization on tiny edge devices. Applying such concepts, such as container technology, to tiny IoT devices, if further enhances the portability and versatility of such devices.

Lastly, we provide a modular toolchain container, containing all tools required for embedded Linux, that can be used to cross-compile Linux or $\mu$Clinux kernel, and evaluate their correctness with QEMU.
\chapter*{Abstract}
\addcontentsline{toc}{chapter}{Abstract}

\selectlanguage{german}
W\"ahrend das Internet der Dinge (Internet of Things, IoT) in Bezug auf die Anzahl der Ger\"ate und die Anwendungsf\"alle ein schnelles Wachstum verzeichnet, mangelt es in diesem Bereich stark an Standardisierung. Mit heterogenen Edge-Ger\"aten, sogenannten Mikrocontrollern (MCUs), und einer Vielzahl von Betriebssystemen (OS) zur Auswahl, leidet die Interoperabilit\"at. Das Ziel dieser Arbeit ist es, den Kernel des Open-Source-Betriebssystems Linux auf kleine MCUs zu portieren. Auf diese Weise wird die Hardware von der Anwendungsschicht abstrahiert und somit die dringend ben\"otigte Standardisierung im IoT-\"Okosystem geschaffen. Dies wurde erreicht, indem die richtige Toolchain gefunden und der Linux- und $\mu$Clinux-Kernel mit Hilfe von Tools wie Buildroot kompiliert wurde.  Anschliessend wurden die kompilierten Distributionen mit QEMU getestet und auf das STM32L476G-Eval Board bzw. ESP-EYE portiert. Dar\"uber hinaus wurde ein anderer Ansatz mit JuiceVM, einer virtuellen RISC-V-Maschine, auf der Linux l\"auft, erprobt.


\selectlanguage{english}

As the Internet-of-Things (IoT) see rapid growth, in device numbers and use cases, standardization is very much lacking in the field. With heterogeneous edge devices, or so-called microcontrollers (MCUs), and a variety of operating systems (OS) to choose from, interoperability is suffering. The goal of this thesis is to port the kernel of the open-source operating system Linux onto tiny MCUs. By doing so abstracting the hardware from the application layer, and therefore providing much-needed standardization in the IoT ecosystem. This was achieved by finding the correct toolchain, and compiling the Linux and $\mu$Clinux kernel with the help of tools such as Buildroot.  Subsequently, the compiled distributions were tested with QEMU and ported to STM32L476G-Eval board and ESP-EYE respectively. Additionally, a different approach with JuiceVM, a RISC-V virtual machine, running Linux was explored.
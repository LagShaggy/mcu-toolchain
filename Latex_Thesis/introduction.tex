\chapter{Introduction}

\section{Motivation}

In the previous two decades, the development and use of sensing- and connectivity-enabling electronic gadgets has steadily increased, in some areas substituting conventional physical devices~\cite{schiller}. With a central focus on interconnectedness, the appropriately named, Internet of-Things (IoT) refers to the billions physical gadgets connected to the internet of throughout the world, all gathering, and more importantly, sharing massive amounts of data. IoT has become an integral part of the lives of billions of people worldwide, not only due to the sheer number of connected devices and potential use cases~\cite{iot-size} but also due to the diversity and variety of IoT solutions~\cite{banafa2016iot}. Many people believe that IoT is the essential development of the twenty-first century, because it affects almost every industry, from healthcare to transportation. However, as the world around us becomes increasingly linked, protecting these resource-constrained devices has become critical.

Linux runs on some microcontrollers, such as the Raspberry Pi (RPI) family. For example, RPI 3B is a small computer equipped with computing essentials, e.g., the Advanced RISC Machines (ARM) Cortex central processing unit (CPU), Random Access Memory (RAM), but not necessarily of the latest generation, having minor energy requirements. As an example, instead of a hard drive, an RPI is equipped with a flash memory card, on which an Operating System (OS) may be installed. It also offers Universal Serial Bus (USB) connectors, a video output, and a Wireless Fidelity (Wi-Fi) adapter. As RPI is a small computer, a regular general-purpose OS such as Linux-based Ubuntu for the ARM architecture can also be supported. However, Raspberry Pi OS, previously known as Raspbian, is a typical distribution of choice for the Raspberry Pi device family. Linux is an excellent success on Raspberry Pi as it simplifies the development of microcontroller applications, as a regular operating can be used with which users are already familiar. Currently, a new generation of microcontrollers is being introduced into the market. For example, the ESP32 device family is based on the dual-core Reduced Instruction Set Computer (RISC)-based Tensilica LX6 processor with a maximum frequency of 240 MHz, 8 MB PSRAM, and 4 MB flash seems to be a great choice to run Linux on those devices as well. Linux was first developed for the Complex instruction Set Computer (CISC)-based Intel 386 (i386) architecture in 1991. Back then, the typical CPU clock speed of the i386 system was between 12 MHz to 40 MHz, while the typical computer was equipped with several megabytes of RAM (e.g., 4 MB). As ESP32 already exceeds the specification of early i386 systems, it seems to be that porting Linux for those devices shall be possible. 8 mega byte (MB) pseudo static RAM (PSRAM) on ESP32-WROVER-IE shall be satisfactory to run the kernel, uclibc, and essential binaries. 

The cheapest development board for ESP32-WROVER-IE costs around 10 CHF. However, when one does not need a development board, an ESP32-WROVER-IE costs 3 CHF. Regular RPI 0 devices, which already contain an ARM Cortex CPU, cost 22-24 CHF, while an RPI 3 costs around 38 CHF. The cost reduction from RPI 3 to Linux-capable ARM-based RPI 0 is already 42\%, and the further cost reduction from RPI (Zero development board) to ESP32-WROVERIE would be another 54\%. Running Linux on regular ESP32-WROVER-IE (i.e., not with a development board) would mean a cost reduction of 92\% in comparison to an RPI 3 device. This is a massive incentive to port a Linux-based distribution towards the new family of devices. The migration of Linux on ESP32 should be possible as there is already a Linux kernel project supporting the kernel execution on the Tensilica LX6 processor family. Furthermore, there are emulators of the RISC-V platform for Tensilica LX6 processors, allowing for executing the code compiled for RISC-V architectures displaying poor performance~\cite{juiceVM}.

RISC-V is another open standard instruction set architecture (ISA) that was first released in 2010 and is based on RISC (Reduced Instruction Set Computer) principles. A layered security method that employs a Trusted Execution Environment (TEE) provided in the RISC-V architecture is a gamechanger in the IoT industry. Unlike most other ISA designs, RISC-V is available under open-source that does not require license fees. RISC-V hardware is available from several companies. Opensource operating systems with RISC-V support are available, and many major software toolchains support the instruction set. Furthermore, there is a Linux kernel available for the RISC-V processor family. As an example, the Espressif ESP32-C3 is a single-core, 32-bit, RISC-V-based MCU with 400KB of SRAM and a 160 MHz clock speed. It includes 2.4 GHz Wi-Fi and Bluetooth 5 (LE) with built-in long-range capability.


\section{Description of Work}

The focus of this thesis is to lay a foundation for porting the open source software (OSS) Linux kernel to tiny microcontrollers, to condense a pratical guide for future studies, which people can use, and upon which they can build. Due to to sheer amount of IoT devices paired with I/O periferals such as sensors, and the endless configurations of the Linux kernel, the combinations appear infinite. To set formal footing in the standardization process of microcontroller units (MCU) with the Linux kernel, this thesis explores the vast amount of tools used for embedding and evaluating the Linux kernel on MCUs.

\section{Thesis Outline}



